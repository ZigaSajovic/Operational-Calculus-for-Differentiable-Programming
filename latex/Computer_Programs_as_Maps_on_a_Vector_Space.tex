\section{Computer Programs as Maps on a Vector Space}

We will model computer programs as maps on a vector space. If
we only focus on the real valued variables (of type \texttt{float} or
\texttt{double}), the state of the memory can be seen as a high
dimensional vector\footnote{We assume the variables of interest to be of type \texttt{float} for
  simplicity. Theoretically any field can be used instead of $\RR$.}. 
A set of all the possible states of the program's memory,
can be modeled by a finite dimensional real vector space $\VV\equiv \RR^n$. We
will call $\VV$ the \emph{memory space of the program}. The effect of a computer
program on its memory space $\VV$, can be described by a map
\begin{equation}
  \label{eq:map}
  P:\VV\to \VV.
\end{equation}
A programming space is a space of maps $\VV\to\VV$ that can be implemented as a
program in a specific programming language. 
\begin{definition}[Euclidean machine] The tuple $(\VV,\F)$ is an Euclidean machine, where
  \begin{itemize}
  \item
  $\VV$ is a finite dimensional vector space over a complete field $K$, serving
  as memory\footnote{In most applications the field $K$ will
    be $\RR$}
  \item
  $\F< \VV^\VV$ is a subspace of the space of maps $\VV\to \VV$, called the \emph{programming space}, serving as actions on the memory.
  \end{itemize}  
\end{definition}

At first glance, the \emph{Euclidean machine} seems like a description of functional programming, with its compositions inherited from $\F$. An intended impression, as we wish for the \emph{Euclidean machine} to build on its elegance. But note that in the coming section an additional restriction is imposed on $\F$; that of its elements being differentiable.


