\section{Differentiable Maps and Programs}

To define differentiable programs, let us first recall some
definitions from multivariate calculus.
\begin{definition}[Derivative]
  Let $V,U$ be Banach spaces. A map $P:V\to U$ is differentiable at a point
  $\x\in V$, if there exists a linear bounded operator $TP_\x:V\to U$ such that
  \begin{equation}
    \label{eq:frechet}
    \lim_{\h\to 0}\frac{\|P(\x+\h)-P(\x)-TP_\x(\h)\|}{\|\h\|} = 0.
  \end{equation}
  The map $TP_\x$ is called the \emph{Fréchet derivative} of the map $P$ at the
  point $\x$.
\end{definition}
For maps $\RR^n\to \RR^m$ Fréchet derivative can be expressed by multiplication
of vector $\h$ by the Jacobi matrix $\mathbf{J}_{P,\x}$ of partial
derivatives of the components of the map $P$
\begin{equation*}
  T_\x P(\h) = \mathbf{J}_{P,\x}\cdot \h.
\end{equation*}

We assume for the remainder of this section that the map $P:V\to U$ is
differentiable for all $\x\in V$. The derivative defines a map from $V$ to
linear bounded maps from $V$ to $U$. We further assume $U$ and $V$ are finite
dimensional. Then the space of linear maps from $V$ to $U$ is isomorphic to the
tensor product $U\otimes V^*$, where the isomorphism is given by the
tensor contraction, sending a simple tensor $\uu\otimes f\in U\otimes
V^*$ to a linear map
 \begin{equation}
   \label{eq:lin_tenzor}
   \uu\otimes f:\x \mapsto f(\x)\cdot \uu.
 \end{equation}
The derivative defines a map
\begin{eqnarray}
  \label{eq:odvod_preslikava}
  \D P&:& V\to U\otimes V^*\\
  \D P&:& \x \mapsto T_\x P.
\end{eqnarray}
One can consider the differentiability of the derivative itself $\D P$ by
looking at it as a map \eqref{eq:odvod_preslikava}. This leads to the definition
of the higher derivatives.
\begin{definition}[Higher derivatives]
  \label{def:higher_derivatives}
  Let $P:V\to U$ be a map from the vector space $V$ to the vector space $U$. 
The derivative $\D^k P$ of order $k$ of the map $P$ is the map
\begin{eqnarray}\label{eq:partial}
    \label{eq:visji_odvod}
    \D^kP&:&V\to U\otimes (V^*)^{\otimes k}\\
    \D^kP&:&\x\mapsto T_\x\left( \D^{k-1}P \right)
  \end{eqnarray}
\end{definition} 
\begin{remark}
  For the sake of clarity, we assumed in the definition above, that the map $P$
  as well as all its derivatives are differentiable at all points $\x$. If this
  is not the case, 
  definitions above can be done locally, which would introduce mostly technical
  difficulties. 
\end{remark}
Let $\e_1,\ldots,\e_n$ be a basis of $U$ and $x_1,\ldots x_m$ the basis of
$V^*$. Denote by $P_i=x_i\circ P$ the $i-th$ component of the map
$P$ according to the basis $\{\e_i\}$ of $U$.
Then $\D^kP$  can be defined in terms of
directional (partial) derivatives by the formula
\begin{equation}\label{eq:d}
  \partial^kP=\sum_{\forall_{i,\alpha}}\frac{\partial^k P_i}{\partial
      x_{\alpha_1}\ldots \partial x_{\alpha_k}}\e_i\otimes
    dx_{\alpha_1}\otimes\ldots \otimes dx_{\alpha_k}.
\end{equation}

\subsection{Differentiable Programs}

We want to be able to represent the derivatives of a computer program in an
\emph{Euclidean machine} again as a program in the same \emph{Euclidean machine}. We define three subspaces of the virtual memory space $\VV$, that
describe how different parts of the memory influence the final result of the
program.   

Denote by $\e_1,\ldots \e_n$ a standard basis of the memory space $\VV$ and by
$x_1,\ldots x_n$ the dual basis of $\VV^*$. The functions $x_i$ are coordinate
functions on $\VV$ and correspond to individual locations(variables) in the
program memory.

\begin{definition}
  For each program $P$ in the programming space $\F<\VV^\VV$,
  we define the \emph{input} or \emph{parameter space} $I_P<\VV$ and the
  \emph{output space} $O_P<\VV$ to be the minimal vector sub-spaces spanned by
  the standard basis vectors, such that the map $P_e$, defined by the following
  commutative diagram 
\begin{equation} 
    \label{eq:induced_map}
\begin{tikzcd}
  \VV \arrow{r}{P} & 
  \VV \arrow{d}{\mathrm{pr}_{O_P}}\\
  I_P \arrow[hook]{u}{\vec{i}\mapsto \vec{i}+\vec{f}} 
  \arrow{r}{P_e}& O_P
\end{tikzcd}
  \end{equation}
does not depend of the choice of the element 
$\vec{f}\in F_P=(I_P+O_P)^\perp$.

The space $F_P=(I_P+O_P)^\perp$ is called \emph{free space} of the program $P$.
\end{definition}

The variables $x_i$ corresponding to standard
basis vectors spanning the parameter, output and free space are called
\emph{paramters} or \emph{input variables}, \emph{output variables} and
\emph{free variables} correspondingly. Free variables are those that are
left intact by the program and have no influence on the final result other than
their value itself. The output of the program depends only on the values
of the input variables and consists of variables that have changed during
the program. Input parameters and output values might overlap. 

The map $P_e$ is called the \emph{effective map} of the program $P$ and
describes the actual effect of the program $P$ on the memory,
ignoring the free memory. 

The derivative of the effective map is of interest, when we speak about
differentiability of computer programs. 
\begin{definition}[Automatically differentiable programs]
  \label{def:program_derivative}
  A program $P:\VV\to \VV$ is \emph{automatically differentiable} if there exist
  an embedding of the space $O_P\otimes I_P^*$ into the free space $F_P$, and a program $(1+\D P):\VV\to \VV$,
  such that its effective map is the map
  \begin{equation}
    \label{eq:program_derivative}
    P_e\oplus \D P_e:I_P\rightarrow O_P\oplus (O_P\otimes I^*).
  \end{equation}
  A program $P:\VV\to \VV$ is \emph{automatically differentiable of order $k$}
  if there exist a program $\sumd_k P:\VV\to \VV$,
  such that its effective map is the map
  \begin{equation}
    \label{eq:program_derivative_higher}
    P_e\oplus \D P_e\oplus \ldots \D^k P_e:I_P\rightarrow O_P\oplus \left(O_P\otimes I^*\right)\oplus\ldots \left( O_P\otimes \left( I_p^*\right)^{k\otimes} \right).
  \end{equation}
\end{definition}

If a program $P:\VV\to \VV$ is automatically differentiable then it is also
differentiable as a map $\VV\to\VV$. However only the derivative of program's
effective map can be implemented as a program, since the memory space is limited to $\VV$. 
To be able to differentiate a program to the $k$-th order, we have to calculate
and save all the derivatives of the orders $k$ and less.


