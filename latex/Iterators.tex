 \section{Iterators}
  

Let $\mathcal{I}_p$ be the monoid generated by $p\in\dP:V\to V$ under composition $\circ$
  
  \begin{equation}\label{eq:iter_def}
  \mathcal{I}_p=\{p^n:V\to V;\quad p(a)=a\},
  \end{equation}
 with some fixed point $a\in V$. We than turn towards analysing the structure of \eqref{eq:iter_def} in relation to $n$, the number of iterations.

Let $h$ be the map defined with the eigen equation \cite{CompoOper}
  \begin{equation}\label{eq:kh}
  h(p(x))=\lambda h(x)
  \end{equation}
   \begin{equation}
   h(a)=0
   \end{equation}
It is clear that the action of $h$ on $p$ is such, 
  \begin{equation}
  h(p^n(x))=\lambda^nh(x)
  \end{equation}
that in the image of $h$, iterations of $p$ become multiplication with the eigen value $\lambda$. We can express the eigen value by differentiating $\eqref{eq:kh}$ at the fixed point $a$,
$$\D h(p(a))\D p(a)=\lambda\D h(a)\implies\D p(a)=\lambda.$$

The derived equips us to inquire about the rate of change of the values of a program $p$ in relation to $n$, the number of iterations. Lets define iterating velocity as
  \begin{equation}
  v(p^n)=\D_np^n(x)
  \end{equation}
Of course
  \begin{equation}
  v(p^n(a))=0
  \end{equation}
the iterating velocity at the fixed point $a$ is constantly zero, which is deduced from the $\eqref{eq:kh}$ and reassures our intuition. Next, we introduce a change of variables $\lambda= e^\nu$ for mathematical convenience and proceed towards computing the iterating velocity.
  $$\D_nh(p^n)=\D_n(e^{\nu n}h(x))$$
  $$\implies$$
  $$\D h(p^n(x))\D_np^n(x)=\nu e^{\nu n}h(x) \land e^{\nu n}h(x)=h(p^n(x))$$
  $$\implies$$
  \begin{equation}\label{eq:iter_vel}
  v=\nu(\D h)^{-1}h
  \end{equation}
The iterating velocity
\footnote{Higher derivatives can be derived by induction.}
\eqref{eq:iter_vel} can be used to study the importance of future iterations for the accumulating result, and aid with decisions on early stopping. Furthermore, the above indicates a subclass of programs for which the Halting problem can be analysed through the study of convergence and fixed points.

The computation of the eigen map $h$ \eqref{eq:kh} was solved by Bridges \cite{bridges2016solution} for any $p$ with a power series representation. This result is extended to tensor series by the isomorphism to their quotient. Hence, as we can expand any $p\in\dP$ into a tensor series by the use of the operator $e^\D$, the result also holds for any $p\in\dP$ by Theorem $\ref{izr:e^d}$. 


\subsection{Example of ReduceSum in the Language of Operational Calculus}

As a demonstration of the algebraic power over analytic conclusions inherent to our model, we examine the functional \emph{ReduceSum}, and derive its explicit form as a function of $n$, the number of its iterations, or upper bound, with special interest in the rate of change of the functional in relation to $n$. 

Let $\mathcal{S}^n$ denote the operator, that performs a linear shift of a program $p$ in the direction $\vv$, from its initial point $\vv_0$.
   \begin{equation}\label{eq:sn_oper}
   \mathcal{S}^n:\dP(\vv_0)\to \dP(\vv_0+n\vv),
   \end{equation}  
and let $\mathcal{S}$, for a specific $\vv_0, \vv\in V$, denote the group generated by \eqref{eq:sn_oper} under composition and addition. By Theorem $\ref{izr:e^d}$ we have
   $$e^{h\D}\dP(\vv_0)(v)=\dP(\vv_0+h\vv)\implies \mathcal{S}^h=e^{h\D},$$
and thus clearly $\mathcal{S}^n\cdot\mathcal{S}^m=\mathcal{S}^{n+m}$ and $(\mathcal{S}^n+\mathcal{S}^m)(\dP)=\mathcal{S}^n(\dP)+\mathcal{S}^m(\dP)$, which we use to define the $n$-th reduction as
$$\mathcal{R}_+^n=(1+\mathcal{S}+\mathcal{S}^2+\cdots+\mathcal{S}^n),$$
that results in
   $$\mathcal{R}^n_+(\dP)(v_0)=\sum\limits_{h=0}^{n}\dP(v_0+hv)$$
upon application.


