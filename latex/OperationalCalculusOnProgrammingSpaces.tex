\documentclass{article}

\usepackage[english]{babel}
\usepackage{amsmath}
\usepackage{amsfonts}
\usepackage{amsthm}
\usepackage{caption}
\usepackage{bbm}
\usepackage[utf8]{inputenc}
\usepackage[T1]{fontenc}
\usepackage{inconsolata}

\usepackage{tikz}
\usepackage{tikz-cd}
\usetikzlibrary{shapes,arrows}
\usetikzlibrary{positioning}

\newcommand{\bfW}{\mathbf{W}}
\newcommand{\bfw}{\mathbf{w}}
\newcommand{\bfM}{\mathbf{M}}
\newcommand{\bfm}{\mathbf{m}}
\newcommand{\RR}{\mathbb{R}}
\newcommand{\Ss}{\mathcal{S}}
\newcommand{\II}{\mathbb{I}}
\newcommand{\JJ}{\mathbb{J}}
\newcommand{\E}{\mathcal{E}}
\newcommand{\T}{\mathcal{T}}
\newcommand{\VV}{\mathcal{V}}
\newcommand{\MM}{\mathcal{M}}
\newcommand{\NN}{\mathcal{N}}
\newcommand{\e}{\mathbf{e}}
\newcommand{\x}{\mathbf{x}}
\newcommand{\m}{\mathbf{m}}
\newcommand{\h}{\mathbf{h}}
\newcommand{\uu}{\mathbf{u}}
\newcommand{\vv}{\mathbf{v}}
\newcommand{\F}{\mathcal{F}}
\newcommand{\X}{\mathcal{X}}
\newcommand{\dP}{\mathcal{P}}
% operator odvoda
\newcommand{\D}{\partial}
%operator 1 + \D
\newcommand{\DD}{\mathcal{D}}
% operator 1+ \D + \D^2 + ...
\newcommand{\sumd}{\tau}
\newcommand{\Op}{\partial^{\bigoplus}}
\newcommand{\op}[1]{\partial^{#1\bigoplus}}
\DeclareMathOperator{\interior}{int}

\newcommand*{\defeq}{\mathrel{\vcenter{\baselineskip0.5ex \lineskiplimit0pt
                     \hbox{\scriptsize.}\hbox{\scriptsize.}}}%
                     =}


\newtheorem{definition}{Definition}[section]
\newtheorem{claim}{Claim}[section]
\newtheorem{theorem}{Theorem}[section]
\newtheorem{remark}{Remark}[section]
\newtheorem{corollary}{Corollary}[theorem]
\newtheorem{proposition}{Proposition}[section]

\usepackage[sorting=none,backend=bibtex]{biblatex}
\bibliography{../latexCommon/biblio.bib}

\title{Operational calculus on Programming Spaces}
\author{\v{Z}iga Sajovic, Martin Vuk}

\begin{document}

\maketitle

\section{Introduction}


Von Neumann languages do not have useful properties for reasoning about programs. Axiomatic and denotational semantics are precise tools for describing and understanding conventional programs, but they only talk about them and cannot alter their ungainly
properties \cite{backus}. This issue raised by John Backus has partially been addressed by algebraic data types employed by functional programming, where a mapping has been shown between grammars and semirings; for an example of how this algebraic view is used to show an isomorphism between data types, see \cite{7Trees}. Yet due to the lack of inverses (hence the semiring structure) we remain limited in the algebraic manipulations we are allowed to employ; for an explanation via a connection between objects of categories and complex numbers see \cite{complexCat}. This leaves one intrigued, but wanting, as the existence of inverses is what gives algebra its true power -- the ability to solve equations.

In recent times, names like \emph{Differentiable Programming} and \emph{Software $2.0$} have attached themselves to Deep learning, as it has shown itself to be more than a collection of machine learning algorithms; it is emerging as a new programming paradigm. But because the field is still in its youth, most of the advances come as a result of empirical investigations. Yet, as it is founded on rigorous mathematical objects, it offers an opportunity to be formalized as a language. And as it is rooted in tensor algebra, it holds the ability to address the outlined issues, because \emph{unlike von Neumann languages, the language of ordinary algebra is suitable both for stating its laws and for transforming
an equation into its solution, all within the language} \cite{backus}.

%analysis and programming and algebra

\printbibliography

\end{document}