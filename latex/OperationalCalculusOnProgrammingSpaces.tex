\documentclass{article}

\usepackage[english]{babel}
\usepackage{amsmath}
\usepackage{amsfonts}
\usepackage{amsthm}
\usepackage{caption}
\usepackage{bbm}
\usepackage[utf8]{inputenc}
\usepackage[T1]{fontenc}
\usepackage{inconsolata}

\usepackage{tikz}
\usepackage{tikz-cd}
\usetikzlibrary{shapes,arrows}
\usetikzlibrary{positioning}

\newcommand{\bfW}{\mathbf{W}}
\newcommand{\bfw}{\mathbf{w}}
\newcommand{\bfM}{\mathbf{M}}
\newcommand{\bfm}{\mathbf{m}}
\newcommand{\RR}{\mathbb{R}}
\newcommand{\Ss}{\mathcal{S}}
\newcommand{\II}{\mathbb{I}}
\newcommand{\JJ}{\mathbb{J}}
\newcommand{\E}{\mathcal{E}}
\newcommand{\T}{\mathcal{T}}
\newcommand{\VV}{\mathcal{V}}
\newcommand{\MM}{\mathcal{M}}
\newcommand{\NN}{\mathcal{N}}
\newcommand{\e}{\mathbf{e}}
\newcommand{\x}{\mathbf{x}}
\newcommand{\m}{\mathbf{m}}
\newcommand{\h}{\mathbf{h}}
\newcommand{\uu}{\mathbf{u}}
\newcommand{\vv}{\mathbf{v}}
\newcommand{\F}{\mathcal{F}}
\newcommand{\X}{\mathcal{X}}
\newcommand{\dP}{\mathcal{P}}
% operator odvoda
\newcommand{\D}{\partial}
%operator 1 + \D
\newcommand{\DD}{\mathcal{D}}
% operator 1+ \D + \D^2 + ...
\newcommand{\sumd}{\tau}
\newcommand{\Op}{\partial^{\bigoplus}}
\newcommand{\op}[1]{\partial^{#1\bigoplus}}
\DeclareMathOperator{\interior}{int}

\newcommand*{\defeq}{\mathrel{\vcenter{\baselineskip0.5ex \lineskiplimit0pt
                     \hbox{\scriptsize.}\hbox{\scriptsize.}}}%
                     =}


\newtheorem{definition}{Definition}[section]
\newtheorem{claim}{Claim}[section]
\newtheorem{theorem}{Theorem}[section]
\newtheorem{remark}{Remark}[section]
\newtheorem{corollary}{Corollary}[theorem]
\newtheorem{proposition}{Proposition}[section]

\usepackage[sorting=none,backend=bibtex]{biblatex}
\bibliography{../latexCommon/biblio.bib}

\title{Operational Calculus on Programming Spaces}
\author{\v{Z}iga Sajovic, Martin Vuk}

\begin{document}

\maketitle

\section{Introduction}

Von Neumann languages do not have useful properties for reasoning about programs. Axiomatic and denotational semantics are precise tools for describing and understanding conventional programs, but they only talk about them and cannot alter their ungainly
properties \cite{backus}. This issue raised by John Backus has partially been addressed by algebraic data types employed by functional programming, where a mapping has been shown between grammars and semirings; for an example of how this algebraic view is used to show an isomorphism between data types, see \cite{7Trees}. Yet due to the lack of inverses (hence the semiring structure) we remain limited in the algebraic manipulations we are allowed to employ; for an explanation via a connection between objects of categories and complex numbers see \cite{complexCat}. This leaves one intrigued, but wanting, as the existence of inverses is what gives algebra its true power -- the ability to solve equations.

In recent times, names like \emph{Differentiable Programming} and \emph{Software $2.0$} have attached themselves to Deep learning, as it has shown itself to be more than a collection of machine learning algorithms; it is emerging as a new programming paradigm.
But because the field is still in its youth, most of the advances come as a result of empirical investigations.
Yet, as it is founded on rigorous mathematical objects, it offers an opportunity to be formalized as a language.
And as it is rooted in tensor algebra, it holds the ability to address the outlined issues, because \emph{unlike von Neumann languages, the language of ordinary algebra is suitable both for stating its laws and for transforming
an equation into its solution, all within the language} \cite{backus}.
Furthermore, due of its innate relationship with calculus, a language encompassing it would serve as a formal gateway of analysis into programming.

TODO: Finish the introduction, notes in the .tex

%a language in which basic words are neural networks


\section{Computer Programs as Maps on a Vector Space}

We will model computer programs as maps on a vector space. If
we only focus on the real valued variables (of type \texttt{float} or
\texttt{double}), the state of the memory can be seen as a high
dimensional vector\footnote{We assume the variables of interest to be of type \texttt{float} for
  simplicity. Theoretically any field can be used instead of $\RR$.}. 
A set of all the possible states of the program's memory,
can be modeled by a finite dimensional real vector space $\VV\equiv \RR^n$. We
will call $\VV$ the \emph{memory space of the program}. The effect of a computer
program on its memory space $\VV$, can be described by a map
\begin{equation}
  \label{eq:map}
  P:\VV\to \VV.
\end{equation}
A programming space is a space of maps $\VV\to\VV$ that can be implemented as a
program in specific programming language. 
\begin{definition}[Euclidean machine] The tuple $(\VV,\F)$ is an Euclidean machine, where
  \begin{itemize}
  \item
  $\VV$ is a finite dimensional vector space over a complete field $K$, serving
  as memory\footnote{In most applications the field $K$ will
    be $\RR$}
  \item
  $\F< \VV^\VV$ is a subspace of the space of maps $\VV\to \VV$, called \emph{programming space}, serving as actions on the memory.
  \end{itemize}  
\end{definition}

At first glance, the \emph{Euclidean machine} seems like a description of functional programming, with its compositions inherited from $\F$. An intended impression, as we wish for the \emph{Euclidean machine} to build on its elegance. But note that in the coming section an additional restriction is imposed on $\F$; that of its elements being differentiable.

\section{Differentiable Maps and Programs}

To define differentiable programs, let us first recall some
definitions from multivariate calculus.
\begin{definition}[Derivative]
  Let $V,U$ be Banach spaces. A map $P:V\to U$ is differentiable at a point
  $\x\in V$, if there exists a linear bounded operator $TP_\x:V\to U$ such that
  \begin{equation}
    \label{eq:frechet}
    \lim_{\h\to 0}\frac{\|P(\x+\h)-P(\x)-TP_\x(\h)\|}{\|\h\|} = 0.
  \end{equation}
  The map $TP_\x$ is called the \emph{Fréchet derivative} of the map $P$ at the
  point $\x$.
\end{definition}
For maps $\RR^n\to \RR^m$ Fréchet derivative can be expressed by multiplication
of vector $\h$ by the Jacobi matrix $\mathbf{J}_{P,\x}$ of partial
derivatives of the components of the map $P$
\begin{equation*}
  T_\x P(\h) = \mathbf{J}_{P,\x}\cdot \h.
\end{equation*}

We assume for the remainder of this section that the map $P:V\to U$ is
differentiable for all $\x\in V$. The derivative defines a map from $V$ to
linear bounded maps from $V$ to $U$. We further assume $U$ and $V$ are finite
dimensional. Then the space of linear maps from $V$ to $U$ is isomorphic to the
tensor product $U\otimes V^*$, where the isomorphism is given by the
tensor contraction, sending a simple tensor $\uu\otimes f\in U\otimes
V^*$ to a linear map
 \begin{equation}
   \label{eq:lin_tenzor}
   \uu\otimes f:\x \mapsto f(\x)\cdot \uu.
 \end{equation}
The derivative defines a map
\begin{eqnarray}
  \label{eq:odvod_preslikava}
  \D P&:& V\to U\otimes V^*\\
  \D P&:& \x \mapsto T_\x P.
\end{eqnarray}
One can consider the differentiability of the derivative itself $\D P$ by
looking at it as a map \eqref{eq:odvod_preslikava}. This leads to the definition
of the higher derivatives.
\begin{definition}[higher derivatives]
  \label{def:higher_derivatives}
  Let $P:V\to U$ be a map from vector space $V$ to vector space $U$. 
The derivative $\D^k P$ of order $k$ of the map $P$ is the map
\begin{eqnarray}\label{eq:partial}
    \label{eq:visji_odvod}
    \D^kP&:&V\to U\otimes (V^*)^{\otimes k}\\
    \D^kP&:&\x\mapsto T_\x\left( \D^{k-1}P \right)
  \end{eqnarray}
\end{definition} 
\begin{remark}
  For the sake of clarity, we assumed in the definition above, that the map $P$
  as well as all its derivatives are differentiable at all points $\x$. If this
  is not the case, 
  definitions above can be done locally, which would introduce mostly technical
  difficulties. 
\end{remark}
Let $\e_1,\ldots,\e_n$ be a basis of $U$ and $x_1,\ldots x_m$ the basis of
$V^*$. Denote by $P_i=x_i\circ P$ the $i-th$ component of the map
$P$ according to the basis $\{\e_i\}$ of $U$.
Then $\D^kP$  can be defined in terms of
directional (partial) derivatives by the formula
\begin{equation}\label{eq:d}
  \partial^kP=\sum_{\forall_{i,\alpha}}\frac{\partial^k P_i}{\partial
      x_{\alpha_1}\ldots \partial x_{\alpha_k}}\e_i\otimes
    dx_{\alpha_1}\otimes\ldots \otimes dx_{\alpha_k}.
\end{equation}

\subsection{Differentiable Programs}

We want to be able to represent the derivatives of a computer program in an
\emph{Euclidean machine} again as a program in the same \emph{Euclidean machine}. We define three subspaces of the virtual memory space $\VV$, that
describe how different parts of the memory influence the final result of the
program.   

Denote by $\e_1,\ldots \e_n$ a standard basis of the memory space $\VV$ and by
$x_1,\ldots x_n$ the dual basis of $\VV^*$. The functions $x_i$ are coordinate
functions on $\VV$ and correspond to individual locations(variables) in the
program memory.

\begin{definition}
  For each program $P$ in the programming space $\F<\VV^\VV$,
  we define the \emph{input} or \emph{parameter space} $I_P<\VV$ and the
  \emph{output space} $O_P<\VV$ to be the minimal vector sub-spaces spanned by
  the standard basis vectors, such that the map $P_e$, defined by the following
  commutative diagram 
\begin{equation} 
    \label{eq:induced_map}
\begin{tikzcd}
  \VV \arrow{r}{P} & 
  \VV \arrow{d}{\mathrm{pr}_{O_P}}\\
  I_P \arrow[hook]{u}{\vec{i}\mapsto \vec{i}+\vec{f}} 
  \arrow{r}{P_e}& O_P
\end{tikzcd}
  \end{equation}
does not depend of the choice of the element 
$\vec{f}\in F_P=(I_P+O_P)^\perp$.

The space $F_P=(I_P+O_P)^\perp$ is called \emph{free space} of the program $P$.
\end{definition}

The variables $x_i$ corresponding to standard
basis vectors spanning the parameter, output and free space are called
\emph{paramters} or \emph{input variables}, \emph{output variables} and
\emph{free variables} correspondingly. Free variables are those that are
left intact by the program and have no influence on the final result other than
their value itself. The output of the program depends only on the values
of the input variables and consists of variables that have changed during
the program. Input parameters and output values might overlap. 

The map $P_e$ is called the \emph{effective map} of the program $P$ and
describes the actual effect of the program $P$ on the memory,
ignoring the free memory. 

The derivative of the effective map is of interest, when we speak about
differentiability of computer programs. 
\begin{definition}[Automatically differentiable programs]
  \label{def:program_derivative}
  A program $P:\VV\to \VV$ is \emph{automatically differentiable} if there exist
  an embedding of the space $O_P\otimes I_P^*$ into the free space $F_P$, and a program $(1+\D P):\VV\to \VV$,
  such that its effective map is the map
  \begin{equation}
    \label{eq:program_derivative}
    P_e\oplus \D P_e:I_P\rightarrow O_P\oplus (O_P\otimes I^*).
  \end{equation}
  A program $P:\VV\to \VV$ is \emph{automatically differentiable of order $k$}
  if there exist a program $\sumd_k P:\VV\to \VV$,
  such that its effective map is the map
  \begin{equation}
    \label{eq:program_derivative_higher}
    P_e\oplus \D P_e\oplus \ldots \D^k P_e:I_P\rightarrow O_P\oplus \left(O_P\otimes I^*\right)\oplus\ldots \left( O_P\otimes \left( I_p^*\right)^{k\otimes} \right).
  \end{equation}
\end{definition}

If a program $P:\VV\to \VV$ is automatically differentiable then it is also
differentiable as a map $\VV\to\VV$. However only the derivative of program's
effective map can be implemented as a program, since the memory space is limited to $\VV$. 
To be able to differentiate a program to the $k$-th order, we have to calculate
and save all the derivatives of the orders $k$ and less.

\section{Differentiable Programming Spaces}

The memory space of a program is rarely treated as more than a storage. But to endow the \emph{Euclidean machine} with extra structure, this is precisely what to focus on. Loosely speaking, functional programming is described by monoids, as a (multi)linear algebra description of the memory space seems the appropriate step to take in attaining the wanted structure.

\subsection{Memory space}

Motivated by the Definition
\ref{def:program_derivative}, we define
the \emph{memory space} for differentiable programs  as a sequence of vector spaces with
the recursive formula
\begin{eqnarray}
  \VV_0 &=& \VV\\
  \label{eq:universal_space}
  \VV_k &=& \VV_{k-1}+\left(\VV_{k-1}\otimes \VV^*\right).
\end{eqnarray}
Note that the sum is not direct, since some of the subspaces of $\VV_{k-1}$ and
$\VV_{k-1}\otimes \VV^*$ are naturally isomorphic and will be
identified\footnote{The spaces $\VV\otimes(\VV^*)^{\otimes (j+1)}$ and
  $\VV\otimes (\VV^*)^{\otimes j}\otimes \VV^*$ are naturally isomorphic and
  will be identified in the sum.
}.

  The space that satisfies the recursive formula (\ref{eq:universal_space}) is
  \begin{equation}
    \label{eq:k-th-virtual-space}
    \VV_k = \VV\otimes \left(K\oplus \VV^* \oplus (\VV^*\otimes \VV^*)\oplus\ldots
      (\VV^*)^{\otimes k}\right) = \VV\otimes T_k(\VV^*),
  \end{equation}
  where $T_k(\VV^*)$ is a subspace of \emph{tensor algebra} $T(\VV^*)$, consisting of
  linear combinations of tensors of rank less or equal $k$. This construction
  enables us to define all the derivatives as maps with 
  the same domain and codomain $\VV\to \VV\otimes T(\VV^*)$.

  As such, the memory space is a mapping from-and-to itself,
\begin{equation}
\VV_n:\VV\to\VV,
\end{equation}
which is, for an arbitrary $\bfW\in\VV_n$, defined as, 
\begin{equation}
\bfW(\vv)=\bfw_0+\bfw_1\cdot\vv+\cdots+\bfw_n\cdot(\vv)^{\otimes n}\label{eq:Contraction},
\end{equation}
the sum of multiple contractions (where $\bfw_i\in\VV_i$). The expression \eqref{eq:Contraction} will be rigorously defined in section \ref{sec:Vrsta}. With such a construction, the expansions and contractions of the memory space (reminiscent to the breathing of the stack) would hold meaning parallel to storing values; which is what motives the next definition.

\begin{definition}[Virtual memory space]\label{def:VV}
Let $(\VV,\F)$ be an Euclidean machine and let  

\begin{equation}
\VV_\infty = \VV\otimes T(\VV^*) = \VV\oplus
(\VV\otimes\VV^*)\oplus\ldots,\label{eq:virtual-memory}
\end{equation}
where $T(\VV^*)$ is the tensor algebra of the dual space $\VV^*$.
We call $\VV_\infty$ the \emph{virtual memory space} of a \emph{Euclidean machine} $(\VV,\F)$.
\end{definition}
The term virtual memory is used as it is only possible to embed certain subspaces of $\VV_\infty$ into memory space $\VV$, making it similar to
virtual memory as a memory management technique. 

We can extend each program $P:\VV\to \VV$ to the map on
universal memory space $\VV_\infty$ by setting the first component in the direct sum
\eqref{eq:virtual-memory} to $P$, and all other components to zero. Similarly
derivatives $\D^k P$ can be also seen as maps  from $\VV$ to $\VV_\infty$ by
setting $k$-th component in the direct sum \eqref{eq:virtual-memory} to $\D^k P$
and all others to zero.

\subsection{Differentiable Programming Spaces}

Let us define the following function spaces:
 \begin{equation}\label{eq:F_n}
  \F_n=\{f:\VV\to \VV\otimes T_n(\VV^*)\}
 \end{equation}
All of these function spaces can be seen as sub spaces of $\F_\infty=\{f:\VV\to \VV\otimes
T(\VV^*)\}$, since $\VV$ is naturally embedded into $ \VV\otimes T(\VV^*)$. The
Fréchet derivative defines an operator on the space of smooth maps in $\F_\infty$\footnote{The operator $\D$ may be defined partially for other maps as
   well, but we will handle this case later.}. We denote this operator $\D$. The image of any map
 $P:\VV\to \VV$ by operator $\D$ is its first derivative, while the higher order
 derivatives are just powers of operator $\D$ applied to $P$.
 Thus $\D^k$ is a mapping between function spaces $\eqref{eq:F_n}$
 \begin{equation}\label{eq:toFn+k}
 \D^k:\F^n\to\F^{n+k}.
 \end{equation}
 
 
 \begin{definition}[Differentiable programming space]\label{def:dP}
  A \emph{differentiable programming space} $\dP_0$ is any subspace of $\F_0$ such that
  \begin{equation}\label{eq:P}
  \D\dP_0\subset\dP_0\otimes T(\VV^*)
\end{equation}
The space $\dP_n<\F_n$ spanned by $\{\D^k\dP_0;\quad 0\le k\le n\}$ over $K$, is
called a differentiable programming space of order $n$. When all elements of
$\dP_0$ are analytic, we call $\dP_0$ an \emph{analytic programming space}. 
 \end{definition}
The definition of higher order differentiable programming spaces is justified by the following theorem. 
\begin{theorem}[Infinite differentiability]\label{izr:P}
  Any differentiable programming space $\dP_0$ is an
  infinitely differentiable programming space, meaning that
  \begin{equation}\label{eq:P_n}
      \D^k\dP_0\subset\dP_0\otimes T(\VV^*)
    \end{equation}
for any $k\in\mathbb{N}$.
\end{theorem}
\begin{proof} By induction on order $k$. For $k=1$ the claim holds by definition. Assume  $\forall_{P\in\dP_0}$,
  $\D^n\dP_0\subset\dP_0\otimes T(\VV^*)$. Denote by $P_{\alpha,k}^i$ the
  component of the $k$-th derivative for a multiindex $\alpha$  denoting the
  component of $T(\VV^*)$ and an index $i$ denoting the component of $\VV$.
  \begin{equation}\label{eq:inductionStep}
\D^{n+1}P_{\alpha,k}^i=\D(\D^n P^i_\alpha)_k\land(\D^n P^i_\alpha)\in\dP_0\implies \D(\D^n P^i_\alpha)_k\in \dP_0\otimes T(\VV^*)
  \end{equation}
  $$\implies$$
  $$\D^{n+1}\dP_0\subset\dP_0\otimes T(\VV^*)$$
Thus by induction, the claim holds for all $k\in \mathbb{N}$. 
\end{proof}



 \begin{corollary}\label{izr:P_n}
  A differentiable programming space of order $n$, $\dP_n:\VV\to \VV\otimes
  T(\VV^*)$, can be embedded into the tensor 
  product of the function space $\dP_0$ and the space $T_n(\VV^*)$ of
  multi-tensors of order less than equal $n$:
  \begin{equation}
    \label{eq:D_p_embed}
    \dP_n<\dP_0\otimes T_n(\VV^*).
  \end{equation}
 \end{corollary}
 
By taking the limit as $n\to \infty$, we consider 
  
  \begin{equation}
  \label{eq:P_algebra}
        \dP_\infty < \dP_0\otimes \T(\VV^*),
  \end{equation}
where $\T(\VV^*)=\prod_{k=0}^\infty (\VV^*)^{\otimes k}$ is the \emph{tensor series
  algebra}, the algebra of the infinite formal tensor series.\footnote{The
  tensor series algebra is a completion of the tensor algebra $T(\VV^*)$ in suitable topology.}

\subsection{Virtual Tensor Machine}

We propose an abstract computational model, that is capable of constructing differentiable programming spaces. Such a model provides a
framework for analytic study of programs by algebraic means. 

Following from Theorem \ref{thm:infDif}, the tuple $(\VV, \dP_0)$ -- together with the structure of the tensor algebra $T(\VV^*)$ -- is sufficient for constructing differentiable programming spaces $\dP_\infty$, using linear combinations of elements of $\dP_0\otimes T(\VV^*)$. This motivates the following definition.

\begin{definition}[Virtual tensor machine]\label{def:analyticMachine}
The tuple $M=\langle \VV,\dP_0\rangle$ is an analytic, infinitely  differentiable virtual machine, where
   
    \begin{itemize}
    \item
    $\VV$ is a finite dimensional vector space
    \item
    $\VV\otimes \T(\VV^*)$ is the virtual memory space
    \item
    $\dP_0$ is an analytic programming space over $\VV$.
    \end{itemize}
  \end{definition}

\noindent When composing contractions \eqref{eq:Contraction} of the memory with activation functions $\phi\in\dP$, we note that fully connected \emph{tensor networks},
\begin{equation} \label{eq:tenWord}
\NN(v)=\phi_k\circ W_k\circ\cdots\circ\phi_0\circ W_0(v),
\end{equation}
are basic programs in a virtual tensor machine (the vanilla fully connected neural network is  captured by the restriction $\forall_i(W_i\in\VV_1)$). The formulation \eqref{eq:tenWord} is trivially generalized to convolutional models, but is omitted here for brevity.




\printbibliography

\end{document}