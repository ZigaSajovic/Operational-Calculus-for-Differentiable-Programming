%auto-ignore
\section{Conclusions}

In this paper we presented a theoretical model for differentiable programming.
Throughout the course of the paper we have shown the model to be a complete description of differentiable programming.
Furthermore, the innate algebraic structure of the framework supplements the descriptive power of a language with the ability to reason about the programs it implements, by way of operational calculus.
We have demonstrated its place in the evolution of computer science, where languages are to be endowed with algebraic constructs that hold power over analytic properties of the programs they implement.
This results hope to inspire other practitioners of differentiable programming to reach for operational calculus in their own attempts to further the field. 
