%auto-ignore
\begin{abstract}
  In this work we present a theoretical model for differentiable programming. We
  construct an algebraic language that encapsulates formal semantics of
  differentiable programs by way of \emph{operational calculus}. The algebraic
  nature of Operational Calculus can alter the properties of the
  programs that are expressed within the language, and reason about them.

  To this purpose, we develop an \emph{abstract computational model of
    differentiable programs} of arbitrary order. In the model,
  programs are elements of \emph{programming spaces} and are viewed as maps from
  the \emph{virtual memory space} to itself. Virtual memory space is an algebra
  of programs, \emph{an algebraic data structure} one can calculate with.
   
  We define the \emph{operator of differentiation} ($\D$) on programming spaces
  and, using its powers, implement the \emph{general shift operator} and the
  \emph{operator of program composition}. We provide the formula for the
  expansion of a differentiable program into an infinite tensor series in terms
  of the powers of $\D$. We express the operator of program composition in terms
  of the generalized shift operator and $\D$, which implements a differentiable
  composition in the language. We prove that our language enables differentiable
  derivatives of programs by the use of the \emph{order reduction map}.
  
  We demonstrate our models algebraic power over analytic properties of
  differentiable programs by analysing iterators, considering \emph{fractional
    iterations} and their \emph{iterating velocities}, and explicitly solve the special
  case of \emph{ReduceSum}.
 \end{abstract}

