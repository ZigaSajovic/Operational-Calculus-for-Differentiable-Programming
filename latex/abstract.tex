%auto-ignore
\begin{abstract}
  In this work we present a theoretical model for differentiable programming. We
  construct an algebraic language that encapsulates formal semantics of
  differentiable programs by way of \emph{Operational Calculus}. The algebraic
  nature of Operational Calculus can alter the properties of the
  programs that are expressed within the language and transform them into their
  solutions.

  In our model programs are elements of \emph{programming spaces} and viewed as
  maps from the \emph{virtual memory space} to itself. Virtual memory space is
  also an algebra of programs, \emph{an algebraic data structure} one can
  calculate with. We define the \emph{operator of differentiation} ($\D$) on
  programming spaces and, using its powers, implement the \emph{general shift
    operator}. We provide the formula for the expansion of a differentiable
  program into an infinite tensor series in terms of the powers of $\D$ and
  implement a differentiable composition of differentiable programs by
  expressing the \emph{operator of program composition} in terms of the
  generalized shift operator and $\D$. The presented operators serve as
  an abstraction and act as the main components of our language.
  
  We demonstrate our model's usefulness in differentiable programming by using it
  to analyze iterators, deriving \emph{fractional iterations} and their
  \emph{iterating velocities}, and explicitly solve the special
  case of \emph{ReduceSum}.
 \end{abstract}

