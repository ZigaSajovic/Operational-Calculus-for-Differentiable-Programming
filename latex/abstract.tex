%auto-ignore
\begin{abstract}
  In this work we present a theoretical model for differentiable programming. We
  construct an algebraic language that encapsulates formal semantics of
  differentiable programs by way of \emph{Operational Calculus}. The algebraic
  nature of Operational Calculus can alter the properties of the
  programs that are expressed within the language, and reason about them.

 In our model programs are elements of \emph{programming spaces} and viewed
  as maps from the \emph{virtual memory space} to itself. Virtual memory space is
  an algebra of programs, \emph{an algebraic data structure} one can calculate with.
   We define the \emph{operator of differentiation} ($\D$) on programming spaces
  and, using its powers, implement the \emph{general shift operator} and the
  \emph{operator of program composition}. We provide the formula for the
  expansion of a differentiable program into an infinite tensor series in terms
  of the powers of $\D$. We express the operator of program composition in terms
  of the generalized shift operator and $\D$, which implements a differentiable
  composition in the language. Such operators serve as abstractions
  over the tensor series algebra, as main actors in our language.
  
  We demonstrate our models usefulness in differentiable programming by using it
  to analyse iterators, deriving \emph{fractional iterations} and their
  \emph{iterating velocities}, and explicitly solve the special
  case of \emph{ReduceSum}.
 \end{abstract}

