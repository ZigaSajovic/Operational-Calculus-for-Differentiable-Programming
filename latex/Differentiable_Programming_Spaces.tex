\section{Differentiable Programming Spaces}

The memory space of a program is rarely treated as more than a storage. But to endow the \emph{Euclidean machine} with extra structure, this is precisely what to focus on. Loosely speaking, functional programming is described by monoids, and as such a (multi)linear algebra description of the memory space seems the appropriate step to take in attaining the wanted structure.

\subsection{Memory space}

Motivated by the Definition
\ref{def:program_derivative}, we define
the \emph{memory space} for differentiable programs  as a sequence of vector spaces with
the recursive formula
\begin{eqnarray}
  \VV_0 &=& \VV\\
  \label{eq:universal_space}
  \VV_k &=& \VV_{k-1}+\left(\VV_{k-1}\otimes \VV^*\right).
\end{eqnarray}
Note that the sum is not direct, since some of the subspaces of $\VV_{k-1}$ and
$\VV_{k-1}\otimes \VV^*$ are naturally isomorphic and will be
identified\footnote{The spaces $\VV\otimes(\VV^*)^{\otimes (j+1)}$ and
  $\VV\otimes (\VV^*)^{\otimes j}\otimes \VV^*$ are naturally isomorphic and
  will be identified in the sum.
}.

  The space that satisfies the recursive formula (\ref{eq:universal_space}) is
  \begin{equation}
    \label{eq:k-th-virtual-space}
    \VV_k = \VV\otimes \left(K\oplus \VV^* \oplus (\VV^*\otimes \VV^*)\oplus\ldots
      (\VV^*)^{\otimes k}\right) = \VV\otimes T_k(\VV^*),
  \end{equation}
  where $T_k(\VV^*)$ is a subspace of \emph{tensor algebra} $T(\VV^*)$, consisting of
  linear combinations of tensors of rank less or equal $k$. This construction
  enables us to define all the derivatives as maps with 
  the same domain and codomain $\VV\to \VV\otimes T(\VV^*)$.

  As such, the memory space is a mapping from-and-to itself,
\begin{equation}
\VV_n:\VV\to\VV,
\end{equation}
which is, for an arbitrary $\bfW\in\VV_n$, defined as, 
\begin{equation}
\bfW(\vv)=\bfw_0+\bfw_1\cdot\vv+\cdots+\bfw_n\cdot(\vv)^{\otimes n}\label{eq:Contraction},
\end{equation}
the sum of multiple contractions (where $\bfw_i\in\VV_i$). The expression \eqref{eq:Contraction} will be rigorously defined in section \ref{sec:Vrsta}. With such a construction, the expansions and contractions of the memory space (reminiscent to the breathing of the stack) would hold meaning parallel to storing values; which is what motives the next definition.

\begin{definition}[Virtual memory space]\label{def:VV}
Let $(\VV,\F)$ be an Euclidean machine and let  

\begin{equation}
\VV_\infty = \VV\otimes T(\VV^*) = \VV\oplus
(\VV\otimes\VV^*)\oplus\ldots,\label{eq:virtual-memory}
\end{equation}
where $T(\VV^*)$ is the tensor algebra of the dual space $\VV^*$.
We call $\VV_\infty$ the \emph{virtual memory space} of a \emph{Euclidean machine} $(\VV,\F)$.
\end{definition}
The term virtual memory is used as it is only possible to embed certain subspaces of $\VV_\infty$ into memory space $\VV$, making it similar to
virtual memory as a memory management technique. 

We can extend each program $P:\VV\to \VV$ to the map on
universal memory space $\VV_\infty$ by setting the first component in the direct sum
\eqref{eq:virtual-memory} to $P$, and all other components to zero. Similarly
derivatives $\D^k P$ can be also seen as maps  from $\VV$ to $\VV_\infty$ by
setting $k$-th component in the direct sum \eqref{eq:virtual-memory} to $\D^k P$
and all others to zero.

\subsection{Differentiable Programming Spaces}

Let us define the following function spaces:
 \begin{equation}\label{eq:F_n}
  \F_n=\{f:\VV\to \VV\otimes T_n(\VV^*)\}
 \end{equation}
All of these function spaces can be seen as sub spaces of $\F_\infty=\{f:\VV\to \VV\otimes
T(\VV^*)\}$, since $\VV$ is naturally embedded into $ \VV\otimes T(\VV^*)$. The
Fréchet derivative defines an operator on the space of smooth maps in $\F_\infty$\footnote{The operator $\D$ may be defined partially for other maps as
   well, but we will handle this case later.}. We denote this operator $\D$. The image of any map
 $P:\VV\to \VV$ by operator $\D$ is its first derivative, while the higher order
 derivatives are just powers of operator $\D$ applied to $P$.
 Thus $\D^k$ is a mapping between function spaces $\eqref{eq:F_n}$
 \begin{equation}\label{eq:toFn+k}
 \D^k:\F^n\to\F^{n+k}.
 \end{equation}
 
 
 \begin{definition}[Differentiable programming space]\label{def:dP}
  A \emph{differentiable programming space} $\dP_0$ is any subspace of $\F_0$ such that
  \begin{equation}\label{eq:P}
  \D\dP_0\subset\dP_0\otimes T(\VV^*)
\end{equation}
The space $\dP_n<\F_n$ spanned by $\{\D^k\dP_0;\quad 0\le k\le n\}$ over $K$, is
called a differentiable programming space of order $n$. When all elements of
$\dP_0$ are analytic, we call $\dP_0$ an \emph{analytic programming space}. 
 \end{definition}
The definition of higher order differentiable programming spaces is justified by the following theorem. 
\begin{theorem}[Infinite differentiability]\label{izr:P}
  Any differentiable programming space $\dP_0$ is an
  infinitely differentiable programming space, meaning that
  \begin{equation}\label{eq:P_n}
      \D^k\dP_0\subset\dP_0\otimes T(\VV^*)
    \end{equation}
for any $k\in\mathbb{N}$.
\end{theorem}
\begin{proof} By induction on order $k$. For $k=1$ the claim holds by definition. Assume  $\forall_{P\in\dP_0}$,
  $\D^n\dP_0\subset\dP_0\otimes T(\VV^*)$. Denote by $P_{\alpha,k}^i$ the
  component of the $k$-th derivative for a multiindex $\alpha$  denoting the
  component of $T(\VV^*)$ and an index $i$ denoting the component of $\VV$.
  \begin{equation}\label{eq:inductionStep}
\D^{n+1}P_{\alpha,k}^i=\D(\D^n P^i_\alpha)_k\land(\D^n P^i_\alpha)\in\dP_0\implies \D(\D^n P^i_\alpha)_k\in \dP_0\otimes T(\VV^*)
  \end{equation}
  $$\implies$$
  $$\D^{n+1}\dP_0\subset\dP_0\otimes T(\VV^*)$$
Thus by induction, the claim holds for all $k\in \mathbb{N}$. 
\end{proof}



 \begin{corollary}\label{izr:P_n}
  A differentiable programming space of order $n$, $\dP_n:\VV\to \VV\otimes
  T(\VV^*)$, can be embedded into the tensor 
  product of the function space $\dP_0$ and the space $T_n(\VV^*)$ of
  multi-tensors of order less than equal $n$:
  \begin{equation}
    \label{eq:D_p_embed}
    \dP_n<\dP_0\otimes T_n(\VV^*).
  \end{equation}
 \end{corollary}
 
By taking the limit as $n\to \infty$, we consider 
  
  \begin{equation}
  \label{eq:P_algebra}
        \dP_\infty < \dP_0\otimes \T(\VV^*),
  \end{equation}
where $\T(\VV^*)=\prod_{k=0}^\infty (\VV^*)^{\otimes k}$ is the \emph{tensor series
  algebra}, the algebra of the infinite formal tensor series.\footnote{The
  tensor series algebra is a completion of the tensor algebra $T(\VV^*)$ in suitable topology.}

\subsection{Virtual Tensor Machine}

We propose an abstract computational model, that is capable of constructing differentiable programming spaces. Such a model provides a
framework for analytic study of programs by algebraic means. 

Following from Theorem \ref{izr:P}, the tuple $(\VV, \dP_0)$ -- together with the structure of the tensor algebra $T(\VV^*)$ -- is sufficient for constructing differentiable programming spaces $\dP_\infty$, using linear combinations of elements of $\dP_0\otimes T(\VV^*)$. This motivates the following definition.

\begin{definition}[Virtual tensor machine]\label{def:analyticMachine}
The tuple $M=\langle \VV,\dP_0\rangle$ is an analytic, infinitely  differentiable virtual machine, where
   
    \begin{itemize}
    \item
    $\VV$ is a finite dimensional vector space
    \item
    $\VV\otimes \T(\VV^*)$ is the virtual memory space
    \item
    $\dP_0$ is an analytic programming space over $\VV$.
    \end{itemize}
  \end{definition}

\noindent When composing contractions \eqref{eq:Contraction} of the memory with activation functions $\phi\in\dP$, we note that fully connected \emph{tensor networks},
\begin{equation} \label{eq:tenWord}
\NN(v)=\phi_k\circ W_k\circ\cdots\circ\phi_0\circ W_0(v),
\end{equation}
are basic programs in a virtual tensor machine (the vanilla fully connected neural network is  captured by the restriction $\forall_i(W_i\in\VV_1)$). The formulation \eqref{eq:tenWord} is trivially generalized to convolutional models, but is omitted here for brevity.


